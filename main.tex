\documentclass[11pt,a4paper]{article}

% Pakete für deutsche Sprache
\usepackage[ngerman]{babel}
\usepackage[utf8]{inputenc}
\usepackage[T1]{fontenc}

% Weitere wichtige Pakete
\usepackage{graphicx}
\usepackage{listings}
\usepackage{color}
\usepackage{hyperref}
\usepackage{geometry}
\usepackage{fancyhdr}
\usepackage{float}
\usepackage{caption}
\usepackage{subcaption}
\usepackage{setspace}

% Schriftart auf Arial (Helvetica) setzen
\usepackage[scaled]{helvet}
\renewcommand{\familydefault}{\sfdefault}

% Zeilenabstand 1,5-fach
\onehalfspacing

% Blocksatz mit Silbentrennung
\usepackage{microtype}
\sloppy

% Seitenränder entsprechend IHK-Vorgaben
\geometry{
    left=2.5cm,
    right=2.5cm,
    top=4cm,
    bottom=3cm,
    headheight=3cm,
    headsep=0.5cm,
    footskip=1cm
}

% Code-Highlighting Setup
\definecolor{codegreen}{rgb}{0,0.6,0}
\definecolor{codegray}{rgb}{0.5,0.5,0.5}
\definecolor{codepurple}{rgb}{0.58,0,0.82}
\definecolor{backcolour}{rgb}{0.95,0.95,0.92}

\lstdefinestyle{mystyle}{
    backgroundcolor=\color{backcolour},   
    commentstyle=\color{codegreen},
    keywordstyle=\color{magenta},
    numberstyle=\tiny\color{codegray},
    stringstyle=\color{codepurple},
    basicstyle=\ttfamily\footnotesize,
    breakatwhitespace=false,         
    breaklines=true,                 
    captionpos=b,                    
    keepspaces=true,                 
    numbers=left,                    
    numbersep=5pt,                  
    showspaces=false,                
    showstringspaces=false,
    showtabs=false,                  
    tabsize=2
}

\lstset{style=mystyle}

% Kopf- und Fußzeile
\pagestyle{fancy}
\fancyhf{}
\renewcommand{\headrulewidth}{0pt}
\renewcommand{\footrulewidth}{0pt}
\setlength{\headheight}{3cm}
\setlength{\headsep}{0.5cm}
\setlength{\footskip}{1cm}
\lhead{\hspace*{-2.0cm}\vspace*{2cm}\includegraphics[height=2.2cm]{image1.png}}
\cfoot{\thepage}

\begin{document}

% Titelseite
\thispagestyle{empty}

% Logo-Bereich oben
\begin{figure}[H]
    \hspace*{-2.0cm}\vspace*{2cm}\includegraphics[height=2.2cm]{image1.png}
\end{figure}

\vspace{3cm}

% Haupttitel - zentriert
\begin{center}
    {\Huge \textbf{COSPAR Communication System}}
\end{center}

\vfill

% Informationsblock unten
\begin{center}
    \begin{tabular}{ll}
        \textbf{Prüfling:} & Abdullah Diab \\[0.5em]
        \textbf{Geburtsdatum:} & 01.01.2006 \\[0.5em]
        \textbf{Ausbildungsberuf:} & Fachinformatiker für Anwendungsentwicklung \\[0.5em]
        \textbf{Ausbildungsbetrieb:} & AFZ - Aus- und Fortbildungszentrum Bremen \\[0.5em]
        \textbf{Ausbildungsstelle:} & ZARM - Zentrum für angewandte Raumfahrttechnologie und Mikrogravitation \\[0.5em]
        \textbf{Prüflings-Nr.:} & 40016 \\
    \end{tabular}
\end{center}

\vspace{2cm}

\newpage

% Eidesstattliche Erklärung
\section*{Eidesstattliche Erklärung}
\thispagestyle{empty}

Hiermit versichere ich, dass das Projekt mit dem Titel \glqq COSPAR Communication System\grqq{} und die dazugehörige Dokumentation in Einzelarbeit von meiner Person angefertigt wurde. Diese Dokumentation wurde in der Vergangenheit nicht bei Prüfungen zur Bewertung/Begutachtung vorgelegt.

\vspace{3cm}

\noindent
Bremen, den \underline{\hspace{3cm}}

\vspace{2cm}

\noindent
\underline{\hspace{8cm}}\\
Ort, Datum und Unterschrift des Prüfungsteilnehmers

\newpage

% Inhaltsverzeichnis
\pagenumbering{roman}
\setcounter{page}{1}
\tableofcontents
\newpage

% Abbildungsverzeichnis
\thispagestyle{empty}
\listoffigures

% Ab hier beginnt die eigentliche Dokumentation mit arabischen Seitenzahlen
\clearpage
\pagenumbering{arabic}
\setcounter{page}{1}
\section{Einleitung}

\subsection{Der Ausbildungsbetrieb}
% Beschreibung des ZARM als Ausbildungsbetrieb
Das Zentrum für angewandte Raumfahrttechnologie und Mikrogravitation (ZARM) der Universität Bremen ist eine international anerkannte Forschungseinrichtung, die sich auf Grundlagenforschung und angewandte Forschung in der Raumfahrttechnologie und Mikrogravitationsforschung spezialisiert hat. Als interdisziplinäres Zentrum verbindet das ZARM Expertise aus den Bereichen Physik, Ingenieurswissenschaften und Materialwissenschaften.

% Rolle im internationalen Wissenschaftsbetrieb
Das ZARM spielt eine bedeutende Rolle in der internationalen Raumfahrtforschung und unterhält enge Kooperationen mit führenden Raumfahrtagenturen wie der ESA, NASA und DLR. Durch diese Vernetzung ist das ZARM auch aktiv in wissenschaftlichen Organisationen wie COSPAR (Committee on Space Research) vertreten, wo Mitarbeiter wichtige Funktionen als Main Scientific Organizers (MSOs) und Deputy Organizers (DOs) übernehmen.

% Ausbildungskontext  
Im Rahmen der Ausbildung werden praktische IT-Projekte durchgeführt, die reale Problemstellungen aus dem wissenschaftlichen Betrieb aufgreifen und digitale Lösungen entwickeln. Das COSPAR Communication System Projekt ist ein solches Ausbildungsprojekt, das eine konkrete Herausforderung im COSPAR-Organisationsprozess adressiert.

\subsection{Zielsetzung des Projekts}
% TODO: Erkläre hier die Ziele des Communication System Projekts
% - MSOs/DOs effizienter mit ihren Autoren kommunizieren lassen
% - Bulk-E-Mail-Funktionalität implementieren
% - Filterung nach Sessions und Autorentypen ermöglichen
% - Manuelle E-Mail-Erstellung eliminieren

\subsection{Was ist COSPAR?}
% TODO: Erläutere COSPAR
% - Committee on Space Research
% - Internationale Organisation für Weltraumforschung
% - Assemblies und Symposiums
% - Rolle der MSOs und DOs
COSPAR (Committee on Space Research) ist eine internationale wissenschaftliche Organisation, die 1958 vom International Council for Science (ICSU) gegründet wurde und als weltweites Forum für den wissenschaftlichen Austausch in der Weltraumforschung dient. Die Organisation koordiniert internationale Bemühungen in allen Bereichen der Raumfahrtwissenschaften und fördert die Zusammenarbeit zwischen verschiedenen Raumfahrtagenturen und Forschungseinrichtungen weltweit. Das wichtigste Ereignis von COSPAR sind die alle zwei Jahre stattfindenden Scientific Assemblies (Wissenschaftliche Versammlungen), bei denen Tausende von Wissenschaftlern aus aller Welt zusammenkommen, um aktuelle Forschungsergebnisse zu präsentieren und zu diskutieren. Diese Konferenzen sind nach wissenschaftlichen Kommissionen und Panels strukturiert, die verschiedene Bereiche der Weltraumforschung abdecken - von Planetenforschung über Astrophysik bis hin zu Weltraumtechnologie. COSPAR entwickelt wichtige Standards für Planetenschutz, koordiniert Datenformat-Richtlinien und organisiert den strukturierten Prozess der wissenschaftlichen Beitragsverwaltung über webbasierte Systeme. Die komplexe Organisation dieser internationalen Veranstaltungen erfordert ein effizientes Management-System, in dem die Main Scientific Organizers (MSOs) und Deputy Organizers (DOs) eine zentrale Rolle spielen, da sie die Schnittstelle zwischen der COSPAR-Führung und der wissenschaftlichen Gemeinschaft bilden.

\subsection{Die Rolle der MSOs, DOs und Autoren}
% TODO: Erkläre die verschiedenen Benutzerrollen
% - Main Scientific Organizers (MSOs): Hauptverantwortliche für Sessions
% - Deputy Organizers (DOs): Unterstützen MSOs bei der Session-Organisation
% - Autoren: Einreicher von Abstracts (Presenting Authors und Co-Authors)
% - Kommunikationsbeziehungen zwischen den Rollen
% - Warum MSOs/DOs ihre Autoren kontaktieren müssen

\newpage

\section{Anforderungsanalyse}

\subsection{Ist-Analyse}

\subsubsection{Aktuelle Kommunikationsweise der MSOs/DOs}
% TODO: Beschreibe den aktuellen Prozess
% - MSOs öffnen ihr E-Mail-Programm (Outlook, Thunderbird, etc.)
% - Manuelle Eingabe aller Autoren-E-Mail-Adressen
% - Kein Filterungssystem für Sessions oder Autorentypen
% - Zeitaufwändiger Prozess bei vielen Autoren

\subsubsection{Probleme und Schwachstellen}
% TODO: Analysiere die Probleme
% - Tippfehler in E-Mail-Adressen
% - Vergessen von Autoren
% - Zeitverschwendung bei repetitiven Aufgaben
% - Keine Historienführung der gesendeten E-Mails

\subsection{Soll-Analyse}

\subsubsection{Funktionale Anforderungen}
% TODO: Liste die funktionalen Anforderungen auf
% - MSO/DO-Authentifizierung
% - Autorenlisten-Anzeige nach Sessions
% - Filterung nach Autorentyp (Presenting Author, Co-Author)
% - Bulk-E-Mail-Kompositionssystem
% - E-Mail-Vorlagen
% - Kommunikationshistorie

\subsubsection{Nicht-funktionale Anforderungen}
% TODO: Beschreibe nicht-funktionale Anforderungen
% - Performance: Schnelle Filterung auch bei vielen Autoren
% - Benutzerfreundlichkeit: Intuitive Bedienung
% - Sicherheit: Sichere Authentifizierung
% - Kompatibilität: Browser-übergreifende Funktionalität

\subsection{Wirtschaftlichkeitsbetrachtung}
% TODO: Nutzen vs. Aufwand
% - Zeitersparnis für MSOs/DOs
% - Reduzierung von Kommunikationsfehlern
% - Entwicklungsaufwand vs. langfristige Effizienzgewinne

\newpage
\section{Entwurfsphase}

\subsection{Systemarchitektur und Datenfluss}
% TODO: Analysiere und visualisiere die Systemarchitektur
% - Datenfluss zwischen Frontend und Backend
% - Integration mit bestehender COSPAR-Datenbank
% - E-Mail-Versand-Pipeline
% - Autorenfilter-Logik und Datenbankabfragen

\subsection{Technische Architektur}

\subsubsection{Analyse der bestehenden Codebasis}
% TODO: Beschreibe die vorhandene COSPAR-Webseite
% - Welche Technologien werden verwendet?
% - Wie ist die Struktur?
% - Wo muss integriert werden?

\subsubsection{Wahl der Technologien}
% TODO: Begründe deine Technologiewahl
% - Frontend: HTML, CSS, JavaScript
% - Backend: PHP
% - Datenbank: MySQL
% - Warum diese Technologien?

\subsubsection{Anbindung an die bestehende Datenbank}
% TODO: Zeige das Datenbankschema
% - Integration mit bestehenden Tabellen
% - Neue Tabellen für E-Mail-System
% - Relationen zwischen MSOs, Sessions und Autoren

\subsection{Konzeption der Features}

\subsubsection{Autorenfilterung und -auswahl}
% TODO: Konzept der intelligenten Autorenfilterung
% - Session-basierte Filterung (A0.1, A0.2, etc.)
% - Autorentyp-Filter (Presenting Author, Co-Author)
% - Kombinierte Filterkriterien für präzise Auswahl
% - Übersichtliche Ergebnistabelle mit Autoreninformationen
% - "Contact Offers" Funktionalität für gefilterte Gruppen
% - Auswahl-Checkboxen für individuelle Empfängersteuerung

\subsubsection{E-Mail-Erstellung und -versand}
% TODO: Konzept des E-Mail-Kompositionssystems
% - Benutzerfreundlicher Rich-Text-Editor
% - Vordefinierte E-Mail-Vorlagen für häufige Anlässe
% - Automatische Empfängerübernahme aus Filterauswahl
% - Vorschau-Funktion vor dem Versand
% - Massenversand mit Progress-Anzeige
% - Fehlerbehandlung bei Zustellungsproblemen

\subsubsection{Kommunikationsübersicht}
% TODO: Konzept der E-Mail-Historienführung
% - Zentrale Übersicht aller versendeten E-Mails
% - Chronologische und session-basierte Sortierung
% - Schnellsuche nach Empfängern oder Inhalten
% - Status-Anzeige für erfolgreiche/fehlgeschlagene Zustellungen

\subsubsection{E-Mail-Detailansicht}
% TODO: Konzept der detaillierten E-Mail-Informationen
% - Vollständige Anzeige einzelner E-Mail-Nachrichten
% - Detaillierte Empfängerliste mit Zustellungsstatus
% - Versandzeitpunkt und technische Metadaten
% - Möglichkeit zur Weiterleitung oder Wiederholung

\subsection{UI/UX Design}

\subsubsection{Wireframes und Mockups}
% TODO: Füge Wireframes/Mockups ein
% Beispiel für Bildeinfügung:
% \begin{figure}[H]
%     \centering
%     \includegraphics[width=0.8\textwidth]{images/wireframe-communication.png}
%     \caption{Wireframe des Communication Systems}
%     \label{fig:wireframe-communication}
% \end{figure}

\subsubsection{Responsives Designkonzept}
% TODO: Erkläre das Responsive Design
% - Mobile-First-Ansatz
% - Breakpoints für verschiedene Bildschirmgrößen
% - Touch-optimierte Bedienelemente

\subsection{Projektplanung}

\subsubsection{Zeitplan}
% TODO: Erstelle einen Zeitplan
% - Gantt-Diagramm oder Tabelle
% - Meilensteine
% - Puffer eingeplant?

\subsubsection{Meilensteine}
% TODO: Definiere klare Meilensteine
% - M1: Analyse abgeschlossen
% - M2: Design fertig
% - M3: Backend-Grundfunktionen implementiert
% - M4: Frontend-Integration abgeschlossen
% - M5: Testing und Deployment

\newpage
\section{Projektdurchführung}

\subsection{Entwicklungsumgebung einrichten}
% TODO: Beschreibe das Setup
% - XAMPP/WAMP für lokale Entwicklung
% - Code-Editor/IDE (VS Code, PhpStorm, etc.)
% - Versionskontrolle (Git)
% - Browser-Entwicklertools

\subsection{Backend-Implementierung}

\subsubsection{Datenbankzugriffe}
% TODO: Implementierung der Datenbankanbindung
% - PDO für sichere Datenbankverbindungen
% - Abfragen für Autorenfilterung
% - Integration mit bestehenden COSPAR-Tabellen

\subsubsection{MSO/DO-Authentifizierung}
% TODO: Authentication-System
% - Integration in bestehendes COSPAR-Login
% - Session-Management
% - Berechtigungsprüfung für Sessions

\subsubsection{E-Mail-System-Integration}
% TODO: E-Mail-Funktionalität
% - PHPMailer für SMTP-Integration
% - Bulk-E-Mail-Verarbeitung
% - E-Mail-Templates

\subsection{Frontend-Implementierung}
\subsubsection{Filter-Wizard-Interface}
% TODO: Konzept des mehrstufigen Filter-Assistenten
% - Step-by-Step-Filterung mit benutzerfreundlicher Oberfläche
% - Schritt 1: Autorentyp-Auswahl (All Authors, Presenting Authors, Co-Authors)
% - Schritt 2: Präsentationstyp-Filter (All, Oral Presentations, Poster Presentations)
% - Schritt 3: Upload-Status-Filter (All, With Presentation, Without Presentation)
% - Schritt 4: Zusammenfassung und Bestätigung der Filterkriterien
% - Modal-Dialog mit Fortschrittsanzeige und Navigation
% - Zurücksetzen-Funktion für alle Filter

\subsubsection{E-Mail-Kompositions-Interface}
% TODO: E-Mail-Editor-Implementation
% - WYSIWYG-Editor-Integration
% - Template-Auswahl
% - Empfänger-Vorschau

\subsubsection{Kommunikationsübersicht-Interface}

% Email history table display
% Search and sort functions
% Status icons and pagination

\subsubsection{E-Mail-Detail-Interface}

% Modal dialog or separate page for details
% Recipient lists and metadata display
% Action buttons for forwarding/retrying

\subsection{Styling und Design}

\subsubsection{CSS-Implementierung}
% TODO: Erkläre dein Styling
% - Responsive CSS-Design
% - Integration in bestehendes COSPAR-Design
% - Cross-Browser-Kompatibilität

\subsubsection{JavaScript-Funktionalitäten}
% TODO: JS-Implementation
% - Event-Handling für Filter
% - AJAX-Aufrufe für dynamische Inhalte
% - Form-Validierung

\subsection{Testing}

\subsubsection{Funktionalitätstests}
% TODO: Beschreibe Tests
% - Filter-Funktionalität testen
% - E-Mail-Versand testen
% - Cross-Browser-Tests

\subsubsection{Usability Tests}
% TODO: Benutzertests
% - Tests mit MSOs/DOs
% - Feedback-Integration
% - UI/UX-Verbesserungen

\newpage
\section{Projektabschluss}

\subsection{Soll-Ist-Vergleich}
% TODO: Vergleiche Anforderungen mit Ergebnis
% - Tabelle mit Anforderungen und Status
% - Was wurde erreicht?
% - Was nicht und warum?

\subsection{Ausblick und Nutzen für COSPAR}
% TODO: Zukunftsperspektive
% - Wie wird das Communication System genutzt?
% - Welche Verbesserungen bringt es?
% - Zeitersparnis für MSOs/DOs
% - Erweiterungsmöglichkeiten

\subsection{Reflexion}
% TODO: Was hast du gelernt?
% - Technische Erkenntnisse (PHP, JavaScript, MySQL)
% - Projektmanagement
% - Kommunikation mit Stakeholdern

\subsection{Fazit}
% TODO: Zusammenfassung
% - Zielerreichung
% - Persönliche Bewertung
% - Danksagungen

% Anhang
\appendix

\section{Quellcode-Auszüge}
% TODO: Wichtige Code-Beispiele
% - PHP-Backend-Code für Autorenfilterung
% - JavaScript-Code für Frontend-Interaktion
% - SQL-Abfragen für Datenbankzugriffe

\section{Screenshots}
% TODO: Screenshots der fertigen Anwendung
% - Dashboard-Ansicht
% - Autorenfilter-Interface
% - E-Mail-Kompositionssystem
% - Bulk-E-Mail-Interface

\section{Testprotokolle}
% TODO: Dokumentation der Tests
% - Funktionalitätstests
% - Usability-Test-Ergebnisse
% - Performance-Tests

\section{Projektplan (Gantt-Diagramm)}
% TODO: Projektplan als Diagramm

\section{Glossar}
% TODO: Wichtige Begriffe
% - COSPAR, MSO, DO, Abstract, Session, Bulk E-Mail, etc.

% Literaturverzeichnis
\begin{thebibliography}{99}
% TODO: Quellen einfügen
% \bibitem{cospar} COSPAR Website: \url{https://cosparhq.cnes.fr/}
% \bibitem{php} PHP Documentation: \url{https://www.php.net/docs.php}
% \bibitem{mysql} MySQL Documentation: \url{https://dev.mysql.com/doc/}
\end{thebibliography}

\end{document}